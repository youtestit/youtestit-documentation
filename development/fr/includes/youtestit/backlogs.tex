\section{Backlogs}

	\subsection{Les Acteurs:}
		 Les différents acteurs dans le projets sont :
		 \begin{itemize}
		 		\item L'administrateur : C'est la personne en charge de gérer le fonctionnement de l'application. 
		 		il contrôle également le calcul distribué des différents tests Selenium. Il a un rôle technique.
		 		
		 		\item L'utilisateur :  Toute personnes pouvant s'identifier sur l'application est un utilisateur.
		 		
		 		\item Un membre : C'est un utilisateur associé à un projet dans l'application.
		 		
		 		\item Un manageur : C'est un membre qui administre un projet ou plusieurs projets.
		 		
		 		\item Un client : c'est une personne extérieurs des projets de l'application, mais qui à la possibilité de 
		 											visualiser certaines parties de l'application.
		 \end{itemize}

			Les droits sont inclusif, allant de l'utilisateur à l'administrateur (utilisateur - client - membre - manageur - administrateur).


	\subsection{User Story 01:}
	L'utilisateur peut s'identifier sur la page d'accueil. La vérification se fait sur les utilisateurs en base de données ainsi que ceux
	présent dans le LDAP.

	\subsection{User Story 02:}
	Un utilisateur peut s'enregistrer. Un administrateur doit en suite valider l'inscription et associer un rôle à l'utilisateur.
	
	\subsection{User Story 03:}
	Un administrateur peut ajouter, modifier ou supprimer un utilisateur.
	
	\subsection{User Story 04:}
	Un manageur peut ajouter, modifier ou supprimer un projet.
	
	\subsection{User Story 05:}
	Un manageur peut ajouter, modifier ou supprimer un sous-projet.
	
	\subsection{User Story 06:}
	Un manageur peut ajouter ou supprimer un membre à son équipe.
	
	\subsection{User Story 07:}
	un membre peut ajouter, modifier ou supprimer un test Selenium. La suppression est soumise à la validation
	du manager.
	
	\subsection{User Story 08:}
	Un membre peut lancer un test Selenium ou un projet.
	
	
	\subsection{User Story 09:}
	Un membre peut visualiser la page de détail d'un projet.
	
	\subsection{User Story 10:}
	Un membre peut  visualiser, ajouter, modifier ou supprimer un graphique de statistique.
	
	\subsection{User Story 11:}
	Un membre peut visualiser la page de détail d'un test Selenium.
	
	\subsection{User Story 12:}
	Un membre peut récupérer les informations d'un projet ou test via un flux RSS.

	\subsection{User Story 13:}
	un membre peut exporter les informations d'un projet et tests en CSV.
	
	\subsection{User Story 14:}
	un membre peut exporter les informations le cahier de tests au format PDF.
		
	\subsection{User Story 15:}
	un membre peut exporter les informations le cahier de tests au format \LaTeX{}.
		
	\subsection{User Story 16:}
	un membre peut exporter les informations le cahier de tests au format HTML.
		
	\subsection{User Story 17:}
	Un administrateur peut ajouter ou supprimer des membres dans le groupe Administrateurs 
	
	\subsection{User Story 18:}
	Un administrateur peut ajouter ou supprimer des membres dans le groupe Manageurs
	
	\subsection{User Story 19:}
	Un administrateur peut ajouter, modifier ou supprimer des machines pour le calcul distribué.
		
	\subsection{User Story 20:}
	Un administrateur configurer les paramètres de l'application (adresse mail, nombre de calcul simultané, etc...)
		
	\subsection{User Story 21:}
	Les membre ont un dashboard récapitulant l'ensemble de leurs travaux en cours.
	
	\subsection{User Story 22:}
	Le dashboard de l'administrateur contient des blocs supplémentaires, comme pour la disponibilité des machines,
	ou les demandes qu'il lui sont adressées.
		
	\subsection{User Story 23:}
	Un membre hors client peut faire des demandes à l'administrateur:
			\begin{itemize}
				\item demande d'ajout de plugins
				\item demande d'ajout de machine spécifique
				\item etc...
			\end{itemize}

	\subsection{User Story 24:}
	Workflow des demandes à l'administrateur.  Le workflow de l'administrateur doit être relativement
	simple. Il se base sur quelque états :
	
			\begin{itemize}
				\item pris en compte
				\item traité
				\item traité, disponible après redémarrage de l'application
				\item complément d'information
				\item rejeté
				\item clos
			\end{itemize}

	\subsection{User Story 25:}
	Création d'un plugin pour interagir avec un bug tracker (Redmine,JIRA,etc...)
	
	\subsection{Technical User Story 01:}
	Créer la liste des profiles.
	
	\subsection{Technical User Story 02:}
	Concevoir le layout principal de l'application
	
