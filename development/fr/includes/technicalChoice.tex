%===============================================================================
% Technical Choice
%===============================================================================
\newpage{}
\chapter{Les choix techniques}


%*******************************************************************************
\section{Les frameworks :}
%*******************************************************************************

L'application contient beaucoup d'écrans assez complexes, avec des formulaires,
des diagrammes et des composants divers. Pour ce type d'application il est nécessaire
d'avoir un framework qui aide au développement et qui permette une grande productivité.

Le nombre d'utilisateurs simultanés est assez limité , il n'y a donc pas de raison
d'utiliser un framework Stateless\footnote{Framework dont les pages n'ont pas d’état}.

Les frameworks MVC ne permettent pas une intégration simple et performante de composants
graphiques complexes (diagramme par exemple). Les formulaires et l'ajax restent les gros points
faibles de ces frameworks.

Un framework orienté composant correspondrait d'avantage aux divers problèmes techniques. 
Ce type de framework facilite énormément la création de formulaires. Ils ont souvent un 
framework Ajax interne, évitant ainsi l'écriture de code javascript. Ce point est très important
pour la maintenabilité de l'application. Mais le point le plus important de ces frameworks est
la possibilité de découper une application en petites briques réutilisables. Leurs bibliothèques
de composants facilitent la création d'interfaces riches.


Il existe différents frameworks composants :
\begin{itemize}
	\item \bold{Google GWT :}
		Très en vogue, mais comporte un problème majeur : sa productivité. GWT permet d'
		avoir des pages compatibles avec la plus part des navigateurs. Pour cela, GWT génère
		la page entièrement en javascript. Ce  point pose différents problèmes, comme le fait
		qu'il est difficile d'appliquer une charte graphique à GWT. Il nécessite également beaucoup
		de fichiers pour réaliser un composant.
		\\
		
	\item \bold{Tapestry 5 :}
		C'est un très bon framework, axé sur de la convention, limitant la configuration XML.
		Il est léger et performant, mais la prise en main est un peu complexe et peut de développeurs
		connaissent ce framework.
		\\
		
	\item \bold{Wicket :}
		Wicket a un très bon système de templating, qui permet d'intégrer rapidement des pages
		html. Cependant le côté java est rapidement très verbeux. Pour des applications complexes
		cela peut remettre en cause la maintenabilité. 
		\\

	\item \bold{Seam 3 / JSF2 :}
		Seam et JSF sont devenus les standard de la norme JEE 6. Ce framework facilite énormément
		l'intégration de formulaires complexes, et possède une collection conséquente de composants. 
		C'est également l'une des technologies les plus simples pour la création de composant (via les
		composants composites). Les liaisons entre le frontal, le coeur du framework et la persistance
		y sont très bien gérées. Cependant beaucoup de développeurs ont des \textit{a priori} sur cette technologie.
		En effet JSF 1.1 et les EJB 2 ont laissé beaucoup de mauvais souvenirs. Ce framework est également
		conçue pour un environnement JEE. Malgré les idées reçues Seam et JSF sont des frameworks
		extrêmement productifs, simples et performants. 

\end{itemize}


\newpage{}
Le choix se porte donc sur :
\begin{itemize}
	\item \bold{Seam 3 - JEE 6 :}
		pour sa capacité modulaire, sa faible configuration XML, la capacité de gestion des divers 
		couches applicatives. De plus c'est un framework reposant sur les standards de la norme JEE.
		\\
		
	\item \bold{JPA 2 :}	
		Seam a été conçu pour fonctionner avec JPA\footnote{Java Persistence API}. JPA permet la persistance de données, et aide
		au mapping entre le modèle objet et le modèle relationnelle des bases de données. Sa configuration
		est essentiellement basée sur des annotations, limitant la configuration XML. Il est également
		possible de l'associer à Hibernate Search pour les recherches full text ainsi que la
		JSR-303\footnote{Validation par annotations} bean
		validator pour la validation des formulaires. JPA et JSF 2 sont totalement compatibles avec 
		la validation par annotations (JSR-303).
		\\
		
	\item \bold{JAX-RS :} pour les services REST de l'application.
		\\
		
	\item \bold{JSF 2 :}
		Ce framework  permet un templating et un découpage par composants. Cette technologie est très 
		productive pour tout ce qui touche aux formulaires et permet une intégration élégante
		d'Ajax. Le projet Facelets y a été intégré nativement, ce qui permet la conception 
		de Layout et l'héritage entre différents layouts. De plus cette technologie fait
		partie intégrante de la norme JEE 6.
		\\
			    			
	\item \bold{ Richfaces 4 / Primefaces 3 :}
		Ce tandem de bibliothèques de composants est conçu pour fonctionner ensemble. Primefaces
		apporte énormément de composants (diagramme, upload en html 5, etc...). Richfaces apporte
		toute la partie abstraction Ajax, qui permet d'avoir des composants riches sans que l'on ait 
		besoin d'écrire de code javascript. La maintenabilité en est considérablement améliorée. 
		\\
				
	\item \bold{PostgreSQL :}	 C'est la base de données Open Source la plus performante.
		\\
			
	\item \bold{JBoss 7 :}
		JBoss ont fait énormément d'optimisation sur leur serveur d'application,  principalement
		sur l'utilisation multi-coeurs des processeurs. Il est également un serveur compatible avec la
		norme JEE 6, ainsi qu'avec l'implémentation de référence pour le class loader (JBoss Weld).
		\\

		
	\item \bold{HTML 5 :}
		Pour profiter au maximum des composants de primefaces, l'application se doit d'être en
		HTML 5, de plus il faut respecter un maximum l'ergonomie et l'accessibilité. 
		\\
						
	\item \bold{JBoss cache / Ehcache :}
		Seam permet une optimisation via l'utilisation du cache JBoss. Ce cache permet d'utiliser
		différentes implémentations. Ehcache permet une grande finesse de configuration. Seam
		permet également d'utiliser Ehcache dans les templates JSF.
\end{itemize}



%+++++++++++++++++++++++++++++++++++++++++++++++++++++++++++++++++++++++++++++++
\section{Environnement système :}
%+++++++++++++++++++++++++++++++++++++++++++++++++++++++++++++++++++++++++++++++

\subsection{Packaging}
L'environnement nécessaire au fonctionnement de l'application reste assez classique, 
elle nécessite un serveur d'application ou conteneur de servlets. Toutefois, les applications
JEE 6 peuvent aussi bien être sous forme d'EAR\footnote{Enterprise Application ARchive}   ou de 
WAR\footnote{Web Application ARchive}.

Dans un premier temps \youTestIt{} sera sous forme de WAR, une version EAR sera conçu ultérieurement.


\subsection{Le serveur}
Le serveur de référence est JBoss 7 (7.0.2.Final ou supérieur en packaging \textbf{Everything}). 
Le serveur se doit de rester standard et avec un minimum de configuration (la configuration de la data
source reste nécessaire).

