%===============================================================================
% Technical Environnement
%===============================================================================
\newpage{}
\chapter{Environnement de développement}

%+++++++++++++++++++++++++++++++++++++++++++++++++++++++++++++++++++++++++++++++
\section{Eclipse - indigo}
%+++++++++++++++++++++++++++++++++++++++++++++++++++++++++++++++++++++++++++++++
Le développement de \youTestIt{} est réalisé sous eclipse avec différents plugins afin de standardiser 
le code. Pour toutes contributions, il est préférable  d'utiliser un environnement similaire.

La version utilisée doit être une version supérieure ou égale à \textbf{indigo}. 

Liste des plugins :
\begin{itemize}
	\item JBoss tools : qui permet l'intégration simple de jboss 7 ainsi que des outils pour Seam.
				Le plugin est présent sur le marcket place d'eclipse.
				
	\item Maven Integration for Eclipse : pour la gestion des dépendances.
					Le plugin est présent sur le marketplace d'eclipse.
					
	\item EGit : pour la gestion des sources sous GitHub
					Le plugin est présent sur le marketplace d'eclipse.
					
	\item PMD : pour la vérification du code
					Le plugin s'installe via l'adresse \textbf{http://pmd.sourceforge.net/eclipse } à rajouter dans 
					Help \rightArrow{} Install new software
					
	\item Check style : pour la vérification du code.
					Le plugin est présent sur le marketplace d'eclipse.
					
	\item Grep Console : non obligatoire, mais facilite la lecture de la console.
					Le plugin est présent sur le marketplace d'eclipse.
\end{itemize}

\paragraph{Formatage du code :}
Il est très important de configurer eclipse avec les différents fichiers de formatages
présents dans le projet. 

On retrouve cinq fichiers de configuration présents dans /youtestit/youtestit-code-format:
\begin{itemize}
	\item  \textbf{cleanups.xml :}  Se configure dans : Window \rightArrow{}
				  preferences \rightArrow{} Java \rightArrow{} Code Style \rightArrow{} Clean up

	\item \textbf{codetemplates.xml :}  Se configure dans : Window \rightArrow{}
				 preferences \rightArrow{} Java \rightArrow{} Code Style \rightArrow{} Code template

	\item \textbf{formatter.xml :} Se configure dans : Window \rightArrow{}
				 preferences \rightArrow{} Java \rightArrow{} Code Style \rightArrow{} Formatter

	\item  \textbf{pmd.xml :} Pour tous les projets importés il faut aller dans leurs propriétés  
		(clique droit sur le projet \rightArrow{} Properties ).
		\begin{enumerate}
			\item aller dans le menu PMD
			\item cliquer sur "activer PMD"
			\item il faut ensuite cliquer sur \emph{Utiliser l'ensemble des règles configurées dans un fichier projet}
			\item cliquer sur parcourir pour rechercher le fichier de configuration du projet.
			\item valider le formulaire.
		\end{enumerate}


	\item  \textbf{checkstyle.xml :}
		Pour le configurer il faut suivre les étapes suivantes :
		\begin{enumerate}
				\item aller dans les préférences d'eclipse
				\item aller dans menu Checkstyle
				\item cliquer sur new
				\item une nouvelle fenêtre s'ouvre, il faut choisir \textbf{External Configuration File}
				\item donner un nom à la configuration
				\item cliquer sur \textbf{Browse...} pour aller rechercher le fichier checkstyle.xml du projet \youTestIt{}.
				\item valider le formulaire.
				\item définir la nouvelle configuration comme configuration par défaut (\textbf{Set as Default})
				\item A présent pour chaque projet importé il faut aller sans sa configuration 
							(clique droit sur le projet \rightArrow{} Properties )
				\item Dans le menu Checkstyle  il faut choisir la configuration  \youTestIt{} et valider le formulaire.
		\end{enumerate}
	

	\paragraph{}	
	
	\begin{attention}
		Toute contribution ne respectant pas les conventions de formatage ne sera pas pris en compte.
	\end{attention}
	
\end{itemize}
%+++++++++++++++++++++++++++++++++++++++++++++++++++++++++++++++++++++++++++++++
\section{GitHub}
%+++++++++++++++++++++++++++++++++++++++++++++++++++++++++++++++++++++++++++++++
L'ensemble du code source du projet se trouve sur GitHub. Pour pouvoir contribuer il est
nécessaire d’être inscrit sur le site.  Une fois inscrit, il faut créer et uploader sa clé RSA, pour cela il faut suivre le tutoriel proposé 
par GitHub : \href{http://help.github.com/linux-set-up-git/ }{http://help.github.com/linux-set-up-git}. 

Une fois le compte gitHub configuré, il suffit de faire une demande d'ajout par mail à l'adresse
\textbf{administrator@youtestit.org}, pour que l'on vous rajoute au projet.

\subsection{Configuration eclipse pour GitHub}
Une fois ajouté sur le projet  vous pouvez récupérer les sources et commencer à contribuer. 
Mais avant cela il faut encore configurer eclispe pour avoir un environnement de développement
fonctionnel. 

\begin{enumerate}
	\item Créez un dossier dans votre home (par exemple : /home/yourLogin/git ). Ce dossier contiendra
	 l'ensemble des sources du projet.
	 
	\item Il faut ensuite cloner tous les projets dans ce dossier
	 \lstsetSh{}
	 \lstinputlisting{includes/src/gitClone.sh}
	 
	 \item Le plugin EGit fonctionne très mal avec la librairie SSH d'eclipse, il faut impérativement lui
	 ordonner de travailler avec la librairie système. Pour cela il faut créer un script shell pour le lancement
	 d'éclispe qui va initialiser les bonnes variables d'environnement.
	 
	 \lstsetSh{}
	 \lstinputlisting{includes/src/eclipse.sh}
	On en profite également pour spécifier l'emplacement de maven et du JDK.
	\newpage

	 \item Une fois eclipse démarré on va pouvoir récupérer nos dépôts git. Dans la perspective
	 Git repository exploring, on peut rajouter un dépôt en cliquant sur l'icone avec un + vert.
		\begin{figure}[!h]
     		\begin{center}
			      \includegraphics[width=0.5\textwidth]{gitAdd}
			      \caption{Ajout d'un dépôt Git}
			      \label{gitAdd}
		    \end{center}
		\end{figure}
		
	 \item Une nouvelle fenêtre s'ouvre dans laquelle on va pouvoir aller chercher  l'emplacement du
	 dépôt Git local. Le bouton "search" permet au plugin d'aller récupérer le .git du dépôt. Il ne 
	 reste plus qu'à valider le formulaire.
		\begin{figure}[!h]
     		\begin{center}
			      \includegraphics[width=0.5\textwidth]{gitRepo}
			      \caption{Récupération des dépôts Git}
			      \label{gitAddRepo}
		    \end{center}
		\end{figure}
		\newpage
	 
	 \item Git a besoin d'une connexion SSH avec une clé RSA pour fonctionner. Il faut
	 bien vérifier que la configuration eclipse pour SSH est correcte et qu'elle contient
	 bien la clé RSA.
		\begin{figure}[!h]
     		\begin{center}
			      \includegraphics[width=0.5\textwidth]{sshParams}
			      \caption{Configuration SSH d'eclipse}
			      \label{eclipseSshConfig}
		    \end{center}
		\end{figure}		 
	 
	\item Une fois la configuration terminée, vous pouvez importer les projets en tant que "general project".
	Pour cela il suffit de déplier le dépôt et de faire un clic droit sur le "Working directory".
		\begin{figure}[!h]
     		\begin{center}
			      \includegraphics[width=0.5\textwidth]{importProject}
			      \caption{Importation d'un projet}
			      \label{gitProjectImport}
		    \end{center}
		\end{figure}	
		
	\item Votre environnement est prêt à faire du pull et du push ;)   Pour plus  d'informations, je vous invite à lire la
	documentation du plugin Egit : \href{http://wiki.eclipse.org/EGit/User\_Guide}{http://wiki.eclipse.org/EGit/User\_Guide} . 
 
	 
\end{enumerate}

%+++++++++++++++++++++++++++++++++++++++++++++++++++++++++++++++++++++++++++++++
\section{PostgreSQL 9.1}
%+++++++++++++++++++++++++++++++++++++++++++++++++++++++++++++++++++++++++++++++
\subsection{Installation}
\begin{enumerate}
	\item Télécharger les binaires à cette adresse : http://www.postgresql.org/download/

	\item Les procédures d'installation sur Linux sont disponibles à cette adresse http://www.openscg.com/se/postgresql/packages.jsp

	\item N'oubliez pas d'ajouter /opt/postgres/9.1/bin au \$PATH

	\item Téléchargez postgresql-9.1-901.jdbc4.jar à cette adresse http://www.jarvana.com/jarvana/archive-details/postgresql/postgresql/9.1-901.jdbc4/postgresql-9.1-901.jdbc4.jar 
et collez le dans ce dossier <chemin_vers_jboss>/standalone/deployments/

	\item Collez le bloc qui suit après la ligne 107 du fichier <chemin_vers_jboss>/standalone/configuration/standalone.xml 
	 \lstsetSh{}
	 \lstinputlisting{includes/src/datasource.xml}
	
\end{enumerate}

\subsection{Configuration}
\begin{enumerate}
	\item Une fois le serveur PostgreSQL démarré, connectez vous en tant qu'utilisateur \textbf{postgres} et importez le script initialisant la base de données :
	 \lstsetSh{}
	 \lstinputlisting{includes/src/postgreSQL.sh}

	\item Retournez sur Eclipse et exécutez le script Ant \textbf{pdeploy} du projet youtestit.

	\item Vous pouvez démarrer votre serveur JBoss et vous rendre à l'adresse suivante : http://localhost:8080/youtestit
\end{enumerate}


%+++++++++++++++++++++++++++++++++++++++++++++++++++++++++++++++++++++++++++++++
\section{youtestit-documentation}
%+++++++++++++++++++++++++++++++++++++++++++++++++++++++++++++++++++++++++++++++
Le projet \textbf{youtestit-documentation} est celui qui contient l'ensemble de la documentation
technique et fonctionnelle. Les différents fichiers sont écrits en LaTeX afin de permettre un meilleur
découpage et la possibilité d'exporter les documentations dans différents formats (PDF, html, text brut).


Les sources LaTeX sont re-compilées  quotidiennement  via la plateforme d'intégration continue (\href{http://youtestit.org/jenkins/job/youtestit-documentation/}{jenkins}).

La conversion en HTML est faite via un fork du projet tex Konverter (\href{http://www.texconverter.org}{www.texconverter.org}). La base de ce projet est bien
conçue et fonctionne plutôt bien. Cependant certaines fonctionnalités n'étaient pas implémentées,
comme la gestion des sources, ou des footnote, etc... Nous sommes en train d'y remédier et nous
reverserons notre contribution une fois le fork terminé. 

%+++++++++++++++++++++++++++++++++++++++++++++++++++++++++++++++++++++++++++++++
\section{youtestit-website}	
%+++++++++++++++++++++++++++++++++++++++++++++++++++++++++++++++++++++++++++++++
Le projet Youtestit-website est la partie web présente à l'url \href{http://www.youtestit.org}{www.youtestit.org}.
A l'heure actuelle le site est constitué de pages web statiques. Nous étudions la possibilité de les
remplacer par une solution de CMS OpenSource (surement basée sur \href{http://www.onehippo.com/en/products/cms}{HippoCMS}). 



%+++++++++++++++++++++++++++++++++++++++++++++++++++++++++++++++++++++++++++++++
\section{youtestit}
%+++++++++++++++++++++++++++++++++++++++++++++++++++++++++++++++++++++++++++++++
C'est le dépôt principal du projet, c'est lui qui contient les sources du projet \youTestIt{}. 



%+++++++++++++++++++++++++++++++++++++++++++++++++++++++++++++++++++++++++++++++
%\section{Datasource : base de données}
%+++++++++++++++++++++++++++++++++++++++++++++++++++++++++++++++++++++++++++++++


%\lstsetSql{}
%\lstinputlisting{includes/src/postgresCreate.sql}

%\lstsetXml{}
%\lstinputlisting{includes/src/datasource.xml}


%jdbc:postgresql://localhost:5432/youtestit
%login/password
